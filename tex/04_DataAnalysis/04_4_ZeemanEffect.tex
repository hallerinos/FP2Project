\subsection{Zeeman Effect in Longitudinal Direction at a Split-Up of 1/4}
\label{toc:ZeemanEffecAnalysis}
In order to determine the wavelength of the $\sigma^-$~- and $\sigma^+$~-~component, the Zeeman split-up should be 1/4 of the dispersion zone. 
Therefore the current is set to the specific value for each element according to table~\ref{tab:ObsZeeman}. 
The course of beam contains a quater-wave plate and a polarisation filter, which can be set to \ang{0} and \ang{90} to filter out each one of the circular components. 
At this time it is not possible to determine which one belongs to which angle, this is possible after doing the analysis.
However one expects the $\sigma^-$~-~line to shift to longer wavelengths and the $\sigma^+$~-~line to shift to shorter wavelengths.\\
For each element and each polarisation a measurement with the first five rings is done similarly to chapter~\ref{toc:Wavelength}. 
Again the center position is determined by averaging the ring's left and right position, now with two rings for each measurement.
\subsubsection{Zeeman Effect of Cadmium}
\label{toc:CdZeeman}
The current is set to $I = \SI{9}{\ampere}$ causing a magnetic field of $B = \SI{187.87 \pm 7.83}{\milli\tesla}$.
The following table shows the measured positions for polarisation \ang{0} and \ang{90}, now referring to $\sigma_1$ and $\sigma_2$. 
\begin{table}[ht]
	\centering
	\begin{minipage}{0.48\textwidth}
		\begin{adjustbox}{width=\textwidth}
			\centering
			\begin{tabular}{c c c c c}
				Ring Number		& $x_\text{left}$ [\si{\mm}]	& $x_\text{right}$ [\si{\mm}]	& $r$ [\si{\mm}]	& $r^2$ [\si{\mm\squared}]	\\
				\hline
				1			& \num{9.69 \pm 0.03} 		& \num{10.19 \pm 0.03} 		& \num{1.22 \pm 0.03} 	& \num{1.48 \pm 0.06} 		\\
				2			& \num{8.74 \pm 0.03} 		& \num{9.06 \pm 0.03} 		& \num{2.25 \pm 0.03} 	& \num{5.09 \pm 0.12} 		\\
				3			& \num{8.06 \pm 0.03} 		& \num{8.32 \pm 0.03} 		& \num{2.96 \pm 0.03} 	& \num{8.79 \pm 0.15} 		\\
				4			& \num{7.56 \pm 0.03} 		& \num{7.74 \pm 0.03} 		& \num{3.50 \pm 0.03} 	& \num{12.29 \pm 0.18} 		\\
				5			& \num{7.05 \pm 0.03} 		& \num{7.25 \pm 0.03} 		& \num{4.00 \pm 0.03} 	& \num{16.04 \pm 0.21} 		\\
%				6			& \num{6.67 \pm 0.03} 		& \num{6.86 \pm 0.03} 		& \num{4.39 \pm 0.03} 	& \num{19.27 \pm 0.23} 		\\ 
			\end{tabular}
		\end{adjustbox}
		\subcaption{Polarisation \ang{0} - $\sigma_1$}
	\end{minipage}
	\hfill
	\begin{minipage}{0.48\textwidth}
		\begin{adjustbox}{width=\textwidth}
			\centering
			\begin{tabular}{c c c c c}
				Ring Number		& $x_\text{left}$ [\si{\mm}]	& $x_\text{right}$ [\si{\mm}]	& $r$ [\si{\mm}]	& $r^2$ [\si{\mm\squared}]	\\
				\hline
				1			& \num{10.36 \pm 0.03} 		& \num{11.31 \pm 0.03} 		& \num{0.36 \pm 0.03} 	& \num{0.13 \pm 0.02} 		\\
				2			& \num{9.14 \pm 0.03} 		& \num{9.54 \pm 0.03} 		& \num{1.86 \pm 0.03} 	& \num{3.44 \pm 0.10} 		\\
				3			& \num{8.41 \pm 0.03} 		& \num{8.66 \pm 0.03} 		& \num{2.66 \pm 0.03} 	& \num{7.08 \pm 0.14} 		\\
				4			& \num{7.83 \pm 0.03} 		& \num{8.03 \pm 0.03} 		& \num{3.27 \pm 0.03} 	& \num{10.66 \pm 0.17} 		\\
				5			& \num{7.33 \pm 0.03} 		& \num{7.49 \pm 0.03} 		& \num{3.79 \pm 0.03} 	& \num{14.33 \pm 0.20} 		\\
%				6			& \num{6.87 \pm 0.03} 		& \num{6.99 \pm 0.03} 		& \num{4.27 \pm 0.03} 	& \num{18.19 \pm 0.22} 		\\
			\end{tabular}
		\end{adjustbox}
		\subcaption{Polarisation \ang{90} - $\sigma_2$}
	\end{minipage}
	\caption[Zeeman Splitting of Cd]{Zeeman splitting of cadmium (1/4 dispersion zone) for polarisation \ang{0} and \ang{90}}
\end{table}\\
The center position was determined to be $x_0 = \SI{11.18 \pm 0.01}{\mm}$. 
According to chapter~\ref{toc:WavelengthDetermination} the two wavelenghts are determined using a linar fit, which is shown in figure~\ref{fig:CdShift}.
\begin{figure}[ht]
	\centering
	\includegraphics[width=0.9\textwidth]{Images/CdPol.pdf}
	\caption[Determination of the Shifted Wavelength $\sigma_1$ \& $\sigma_2$ of Cd]{Determination of the shifted wavelength $\sigma_1$ and $\sigma_2$ of cadmium}
	\label{fig:CdShift}
\end{figure}\\
As a result one gets the relevant fit parameters and hence their corresponding wavelengths.
\begin{table}[ht]
	\centering
	\begin{tabular}{c p{2.0cm} c}
	$m_1 = \SI{3.61 \pm 0.03}{\mm\squared}$		& & $m_2 = \SI{3.59 \pm 0.04}{\mm\squared}$	\\
	$\lambda_{\sigma_1} = \SI{604.98 \pm 5.70}{\nm}$	& & $\lambda_{\sigma_2} = \SI{601.64 \pm 7.18}{\nm}$
	\end{tabular}
\end{table}\\
Now the $\sigma_1$~-~component can be assigned to $\sigma^-$, whereas $\sigma_2$ belongs to the $\sigma^+$~-~component. 
Since the $\sigma^-$~-~component has less energy than the $\sigma^+$~-~component, its radius turns out bigger. 
This would be consistent with the assignation, which should be correct, even though the two wavelength overlap within their interval of uncertainty. 

\subsubsection{Zeeman Effect of Tin}
\label{toc:SnZeeman}
In order to measure the shifted spectral lines for tin the current was set to $I = \SI{9.5}{\ampere}$, leading to a magnetic field of $B = \SI{197.81 \pm 8.04}{\milli\tesla}$. 
Again the two lines were seperated using the polarisation filter, the measured data is shown in table~\ref{tab:SnZeeman}.
\begin{table}[ht]
	\centering
	\begin{minipage}{0.48\textwidth}
		\begin{adjustbox}{width=\textwidth}
			\centering
			\begin{tabular}{c c c c c}
				Ring Number		& $x_\text{left}$ [\si{\mm}]	& $x_\text{right}$ [\si{\mm}]	& $r$ [\si{\mm}]	& $r^2$ [\si{\mm\squared}]	\\
				\hline
				1			& \num{10.69 \pm 0.03} 		& \num{11.23 \pm 0.03} 		& \num{0.23 \pm 0.02} 	& \num{0.05 \pm 0.01} 		\\
				2			& \num{9.33 \pm 0.03} 		& \num{9.66 \pm 0.03} 		& \num{1.69 \pm 0.02} 	& \num{2.87 \pm 0.08} 		\\
				3			& \num{8.54 \pm 0.03} 		& \num{8.78 \pm 0.03} 		& \num{2.53 \pm 0.02} 	& \num{6.39 \pm 0.12} 		\\
				4			& \num{7.95 \pm 0.03} 		& \num{8.09 \pm 0.03} 		& \num{3.17 \pm 0.02} 	& \num{10.04 \pm 0.15} 		\\
				5			& \num{7.43 \pm 0.03} 		& \num{7.57 \pm 0.03} 		& \num{3.69 \pm 0.02} 	& \num{13.61 \pm 0.17} 		\\
%				6			& \num{6.95 \pm 0.03} 		& \num{7.11 \pm 0.03} 		& \num{4.16 \pm 0.02} 	& \num{17.30 \pm 0.20} 		\\
			\end{tabular}
		\end{adjustbox}
		\subcaption{Polarisation \ang{0} - $\sigma_1$}
	\end{minipage}
	\hfill
	\begin{minipage}{0.48\textwidth}
		\begin{adjustbox}{width=\textwidth}
			\centering
			\begin{tabular}{c c c c c}
				Ring Number		& $x_\text{left}$ [\si{\mm}]	& $x_\text{right}$ [\si{\mm}]	& $r$ [\si{\mm}]	& $r^2$ [\si{\mm\squared}]	\\
				\hline
				1			& \num{9.89 \pm 0.03} 		& \num{10.50 \pm 0.03} 		& \num{0.99 \pm 0.02} 	& \num{0.99 \pm 0.05} 		\\
				2			& \num{8.91 \pm 0.03} 		& \num{9.20 \pm 0.03} 		& \num{2.13 \pm 0.02} 	& \num{4.55 \pm 0.10} 		\\
				3			& \num{8.16 \pm 0.03} 		& \num{8.50 \pm 0.03} 		& \num{2.86 \pm 0.02} 	& \num{8.17 \pm 0.14} 		\\
				4			& \num{7.70 \pm 0.03} 		& \num{7.88 \pm 0.03} 		& \num{3.40 \pm 0.02} 	& \num{11.55 \pm 0.16} 		\\
				5			& \num{7.23 \pm 0.03} 		& \num{7.36 \pm 0.03} 		& \num{3.89 \pm 0.02} 	& \num{15.16 \pm 0.18} 		\\
%				6			& \num{6.81 \pm 0.03} 		& \num{6.89 \pm 0.03} 		& \num{4.34 \pm 0.02} 	& \num{18.82 \pm 0.21} 		\\	
			\end{tabular}
		\end{adjustbox}
		\subcaption{Polarisation \ang{90} - $\sigma_2$}
	\end{minipage}
	\caption[Zeeman Splitting of Sn]{Zeeman splitting of tin (1/4 dispersion zone) for polarisation \ang{0} and \ang{90}}
	\label{tab:SnZeeman}
\end{table}\\
With a center position of $x_0 = \SI{11.19 \pm 0.01}{\mm}$ the wavelengths were determined via the fits, that are shown in figure~\ref{fig:SnShift}.
\begin{figure}[ht]
	\centering
	\includegraphics[width=0.9\textwidth]{Images/SnPol.pdf}
	\caption[Determination of the Shifted Wavelength $\sigma_1$ \& $\sigma_2$ of Sn]{Determination of the shifted wavelength $\sigma_1$ and $\sigma_2$ of tin}
	\label{fig:SnShift}
\end{figure}\\
As a result, the relevant parameters $m$ and their corresponding wavelengths are listed below.
\begin{table}[ht]
	\centering
	\begin{tabular}{c p{2.0cm} c}
		$m_1 = \SI{3.43 \pm 0.09}{\mm\squared}$		& & $m_2 = \SI{3.54 \pm 0.02}{\mm\squared}$	\\
		$\lambda_{\sigma_1} = \SI{575.16 \pm 1.54}{\nm}$	& & $\lambda_{\sigma_2} = \SI{593.03 \pm 3.51}{\nm}$
	\end{tabular}
\end{table}
In contrast to the measurement of cadmium, the assignation of the components is inverse. 
The $\sigma_1$~-~component can be associated with $\sigma^+$, whereas the $\sigma_2$~-~component belongs to $\sigma^-$, because it turns out smaller.

