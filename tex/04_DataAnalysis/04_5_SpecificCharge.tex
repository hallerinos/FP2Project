\subsection{Determination of the Specific Electron Charge $e/m$}
\label{toc:SpecificCharge}
Using this experimental setup it is possible to determine the specific electron charge $e/m$. 
Equation~\ref{eq:BohrMagneton} shows an expression for $\mu_\text{B}$, that is related to measurable parameters. 
Additionally, the Bohr magneton can be calculated with universal constants. 
\begin{align}
	\mu_\text{B} = h\cdot c \frac{\overline{\delta\nu}}{2B} \overset{!}{=} \frac{e\hbar}{2m_e}
	\label{eq:BohrMagneton}
\end{align}
This yields a relation to calculate $e/m_e$ according to
\begin{align}
	\label{eq:SpecificChargeElectron}
	\frac{e}{m_e} = \frac{2\pi c}{B} \cdot\overline{\delta\nu} \hspace{0.5cm} \text{with} \hspace{0.5cm} \Delta\left(\frac{e}{m_e}\right) = \frac{e}{m_e}\sqrt{\left(\frac{\Delta (\overline{\delta\nu})}{\overline{\delta\nu}}\right)^2 + \left(\frac{\Delta B}{B}\right)^2}\ .
\end{align}
$\overline{\delta\nu}$ was computed individually for cadmium and tin by averaging and is listed in the following table, alongside with the specific magnetic field of 1/4 dispersion zone.
\begin{table}[ht]
	\centering
	\begin{tabular}{c c c c}
	Element		& $\overline{\delta\nu}$ [\si{\mm\squared}]	& $B$ [\si{\milli\tesla}]	& $e/m_e$ [\si[per-mode=fraction]{\coulomb/\kg}]	\\
	\hline
	Cd		& \num{20.63 \pm 4.63}				& \num{187.87 \pm 7.83}		& \num{2.068 \pm 0.471 e11}	\\
	Sn		& \num{14.58 \pm 4.03}				& \num{197.81 \pm 8.04}		& \num{1.388 \pm 0.388 e11}	\\
	\end{tabular}
	\caption[Determination of the Specific Electron Charge]{Determination of the specific electron charge using the Zeeman splitting of cadmium and tin}
	\label{tab:SpecCharge_e}
\end{table}\\
An average of these values yields the final result
\begin{align}
	\frac{e}{m_e} = \SI[per-mode=fraction]{1.728 \pm 0.305 e11}{\coulomb\per\kg}\ .
\end{align}
Compared to the theoretical value of $e/m_e = \SI{1.759 e11}{\coulomb/\kg}$ the experimental value is pretty accurate \cite{SpecificChargeLit}. 
The large uncertainty results from the large errors of $\delta\nu$. 
Nevertheless this value can be considered as an indication of a precise measurement of wavelength differences.
\\
Additionally it is possible to determine the Bohr magneton according to equation~\ref{eq:BohrMag1} and \ref{eq:BohrMag2} with the same precision, since it only diversifies by constants.
\begin{table}[ht]
	\centering
	\begin{tabular}{c p{2.0cm} c}
		$\mu_\text{B,Cd} = \SI[per-mode=fraction]{10.907 \pm 2.485 e-24}{\joule\per\tesla}$	& 	& $\mu_\text{B,Sn} = \SI[per-mode=fraction]{7.319 \pm 2.048 e-24}{\joule\per\tesla}$
	\end{tabular}
\end{table}
The mean, experimentally determined Bohr magneton calculates to $\mu_\text{B} = \SI{9.113 \pm 1.610 e-24}{\joule/\tesla}$, which also is a pretty accurate result compared to the literature value of $\mu_\text{B,lit} = \SI{9.274 e-24}{\joule/\tesla}$ \cite{Povh}. 
Again the large uncertainty propagates from the errors of $\delta\nu$.
