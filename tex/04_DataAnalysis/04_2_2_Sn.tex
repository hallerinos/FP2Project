\subsubsection{Wavelength of Tin}
\label{toc:WavelengthSn}
For the singlet line of the tin lamp the center position of the interference pattern was determined to be $x_0 = \SI{11.20 \pm 0.01}{\mm}$, the values for each measurement are listed in the following table.
\begin{table}[ht]
	\centering
	\begin{tabular}{c c c}
	Ring Number		& $x_\text{left}$ [\si{\mm}]	& $x_\text{right}$ [\si{\mm}]	\\
	\hline
	1			& \num{10.28 \pm 0.03} 		& \num{12.13 \pm 0.03}		 \\
	2			& \num{9.07 \pm 0.03} 		& \num{13.31 \pm 0.03}		 \\
	3			& \num{8.33 \pm 0.03} 		& \num{14.06 \pm 0.03}		 \\
	4			& \num{7.74 \pm 0.03} 		& \num{14.65 \pm 0.03}		 \\
	\end{tabular}
	\caption[Measurement - Center Position Sn]{Measured values to determine the center position for Sn}
	\label{tab:SnCenter}
\end{table}\\
Propagation of uncertainty again yields a small error for the center position. 
The methode of calculation the standard derivation leads to an even smaller error, which shows a precise measurement. 
The values used for the linear fit are listed in table~\ref{tab:SnRings}, the fit itself is shown in figure~\ref{fig:SnWavelengthFit}.
\begin{table}[ht]
	\centering
	\begin{tabular}{c c c c c}
	Ring Number		& $x_\text{left}$ [\si{\mm}]	& $x_\text{right}$ [\si{\mm}]	& $r$ [\si{\mm}]		& $r^2$ [\si{\mm\squared}]	\\
	\hline
	1			& \num{9.97 \pm 0.03} 		& \num{10.37 \pm 0.03} 		& \num{1.03 \pm 0.02} 		& \num{1.05 \pm 0.05} \\
	2			& \num{8.92 \pm 0.03} 		& \num{9.19 \pm 0.03} 		& \num{2.14 \pm 0.02} 		& \num{4.58 \pm 0.10} \\
	3			& \num{8.24 \pm 0.03} 		& \num{8.43 \pm 0.03} 		& \num{2.86 \pm 0.02} 		& \num{8.19 \pm 0.14} \\
	4			& \num{7.68 \pm 0.03} 		& \num{7.84 \pm 0.03} 		& \num{3.44 \pm 0.02} 		& \num{11.81 \pm 0.16} \\
	5			& \num{7.17 \pm 0.03} 		& \num{7.30 \pm 0.03} 		& \num{3.96 \pm 0.02} 		& \num{15.69 \pm 0.19} \\
	6			& \num{6.72 \pm 0.03} 		& \num{6.86 \pm 0.03} 		& \num{4.41 \pm 0.02} 		& \num{19.42 \pm 0.21} \\
	7			& \num{6.31 \pm 0.03} 		& \num{6.47 \pm 0.03} 		& \num{4.81 \pm 0.02} 		& \num{23.10 \pm 0.23} \\
	8			& \num{5.89 \pm 0.03} 		& \num{6.10 \pm 0.03} 		& \num{5.20 \pm 0.02} 		& \num{27.05 \pm 0.25} \\
	9			& \num{5.61 \pm 0.03} 		& \num{5.76 \pm 0.03} 		& \num{5.51 \pm 0.02} 		& \num{30.37 \pm 0.26} \\
	10			& \num{5.31 \pm 0.03} 		& \num{5.45 \pm 0.03} 		& \num{5.82 \pm 0.02} 		& \num{33.83 \pm 0.28} \\
%	11			& \num{5.02 \pm 0.03} 		& \num{5.17 \pm 0.03} 		& \num{6.10 \pm 0.02} 		& \num{37.23 \pm 0.29} \\
	\end{tabular}
	\caption[Measurement - Ring Positions Sn]{Measured values to determine the radii for Sn}
	\label{tab:SnRings}
\end{table}\\
The relevant fit parameter is determined to be $m = \SI{3.68 \pm 0.02}{\mm\squared}$, which results in an experimental wavelength of
\begin{align}
	\lambda_\text{Sn,exp} = \SI{617.66 \pm 5.22}{\nm}
\end{align}
for the singlet tin line. 
Compared to the literature value of $\lambda_\text{Sn,lit} = \SI{636.23}{\nm}$, the measurement shows a better conformity than for the singlet line of cadmium. 
Still the experimental value does not include the literature value in its interval. 
The reasons are the same as for the determination of the cadmium wavelength.
\begin{figure}[ht]
	\centering
	\includegraphics[width=0.7\textwidth]{Images/zeeman3.pdf}
	\caption[Determination of the Wavelength of Sn]{Determination of the wavelength of Sn using a linear fit modell}
	\label{fig:SnWavelengthFit}
\end{figure}\\
With the specified wavelength the order of the center ring can be determined to be
\begin{align}
	n_0 = \num{48894.2 \pm 525.0}\ ,
\end{align}
which yields the order of the first maximum using $\epsilon = \num{0.74 \pm 0.03}$
\begin{align}
	n_1 = \num{48893 \pm 525}\ .
\end{align}
