\subsection{Resolution of the Fabry-P\'erot Interferometer}
\label{toc:ResolutionFPI}
During the observation of the Zeeman effect it was unfortunately neglected to memorise the current value for the splitting to start. 
However it is possible to estimate the resolution of the Fabry-P\'erot interferometer based on the available data, but the result is expected to be far from the actual value.
One can calculate the resolution, starting from equation
\begin{align}
	\begin{split}
	\Delta E	&= h (\nu_{\sigma^+} - \nu_{\sigma^-})\\
			&= h c \frac{\lambda_{\sigma^-}-\lambda_{\sigma^+}}{\lambda_{\sigma^+}\lambda_{\sigma^-}}\\
			&\approx hc \frac{\delta\lambda}{\lambda^2}
	\end{split}
\end{align}
and eliminating $\Delta E = \mu_{\text{B}} \cdot B$ to
\begin{align}
	\delta\lambda	&= \frac{e}{m_e}\cdot\frac{B\lambda^2}{4\pi c}\ ,
\end{align}
which can be resolved to
	\begin{align}
	\begin{split}
	\left(\frac{\lambda}{\delta\lambda}\right)_{\text{exp}}		&= \frac{1}{\frac{e}{m_e}}\frac{4\pi c}{B \lambda}\\
	\Delta\left(\frac{\lambda}{\delta\lambda}\right)_{\text{exp}} 	&= \frac{1}{\frac{e}{m_e}}\frac{4\pi c}{B \lambda}\sqrt{\left(\frac{\Delta \left(\frac{e}{m_e}\right)}{\frac{e}{m_e}}\right)^2 + \left(\frac{\Delta B}{B}\right)^2 +\left(\frac{\Delta \lambda}{\lambda}\right)^2}\ .
	\end{split}
	\end{align}
Applying the calculated values for $B$ and $\lambda$ the resolution is denoted to
	%\begin{align}
	%\left(\frac{\lambda}{\delta\lambda}\right)_{Cd} = \num{0.914 \pm 0.039 e5} \hspace{1.0cm} \left(\frac{\lambda}{\delta\lambda}\right)_{Zn} = \num{0.877 \pm 0.036 e5}\ .
	%\end{align}
	\begin{align}
	\left(\frac{\lambda}{\delta\lambda}\right)_{\text{exp, Cd}} = \num{1.555 \pm 0.036 e5} \hspace{1.0cm} \left(\frac{\lambda}{\delta\lambda}\right)_{\text{exp, Sn}} = \num{2.221 \pm 0.628 e5}\ .
	\end{align}
Theoretically, the finesse $\mathcal{F}$ should be a reference value for the interferometer's resolution, if it is multiplied by $n_0$ (which was determined in chapter~\ref{toc:Wavelength}).
	\begin{align}
	\begin{split}
	\left(\frac{\lambda}{\delta\lambda}\right)_{\text{ref}} 	&= \frac{2 d}{\lambda_\text{lit}} \cdot \frac{\pi \sqrt{R}}{1-R}\\
	\Delta \left(\frac{\lambda}{\delta\lambda}\right)_{\text{ref}}	&= \frac{2 \Delta d}{\lambda_\text{lit}} \cdot \frac{\pi \sqrt{R}}{1-R}
	\end{split}
	\end{align}
	\begin{align}
	\left(\frac{\lambda}{\delta\lambda}\right)_{\text{ref, Cd}} = \num{146.619 \pm 0.971 e5} \hspace{1.0cm} \left(\frac{\lambda}{\delta\lambda}\right)_{\text{ref, Sn}} = \num{148.375 \pm 0.983 e5}
	\end{align}
Therefore the deviation is of $\mathcal{O}(10^2)$.
This will be discussed in chapter~\ref{toc:PlausibilityZeeman}.
