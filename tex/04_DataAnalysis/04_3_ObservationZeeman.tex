\subsection{Observation of the Zeeman Effect in Longitudinal Direction}
\label{toc:ObservationZeeman}
The Zeeman effect was observed while increasing the coil current, which leads to a variation of the magnetic field causing the split-up of the rings. 
For a splitting of 1/4, 1/2 and 3/4 of the dispersion zone the value for the current was noted.\\
Observation in the longitudinal direction only allows to see the $\sigma^-$~- and $\sigma^+$~-~component, but not the $\pi$~-~component (refer to figure~\ref{fig:ZeemanSplitting}). 
For a split-up of 1/4 of the dispersion zone one gets an image with twice as many rings compared to the case without magnetic field. 
The rings are equally spaced and show both of the circular components arranged around the unshifted ring position. 
For a split-up of 3/4 one gets the same picture, but with swapped $\sigma^-$~- and $\sigma^+$~-~lines.\\
If the split-up is 1/2 of the dispersion zone, the picture is the same as for the case without splitting. 
Here both of the circular components overlap at the position of the unshifted spectral lines. 
Table~\ref{tab:ObsZeeman} shows the recorded values for each of the three cases.
\begin{table}[ht]
	\centering
	\begin{tabular}{c c c c c}
	Element	& Fraction of the Dispersion Zone $D$	& $\Delta D$	& $I$ [\si{\ampere}]	& $B$ [\si{\milli\tesla}]	\\
	\hline
		Cd	& \num{0.25}	& \num{0.1}	& \num{9 \pm 0.3}	& \num{187.87 \pm 7.83} 	\\
 			& \num{0.50}	& \num{0.1}	& \num{18.5 \pm 0.3}	& \num{363.66 \pm 15.45} 	\\ 
			& \num{0.75}	& \num{0.1}	& \num{32 \pm 0.3}	& \num{546.76 \pm 40.83} 	\\ 
		Sn	& \num{0.25}	& \num{0.1}	& \num{9.5 \pm 0.3}	& \num{197.81 \pm 8.04} 	\\
 			& \num{0.50}	& \num{0.1}	& \num{17 \pm 0.3}	& \num{337.99 \pm 13.72} 	\\ 
			& \num{0.75}	& \num{0.1}	& \num{32 \pm 0.3}	& \num{546.76 \pm 40.83} 	\\
	\end{tabular}
	\caption[Observation of the Zeeman Effect (Longitudinal Direction)]{Observation of the Zeeman effect in longitudinal direction for three important splittings. The error of the magnetic field was calculated using propagation of uncertainty for the fit parameters and the current.}
	\label{tab:ObsZeeman}
\end{table}\\
The determination of the fraction of the dispersion zone was done by visual judgement, why it holds large uncertainties. 
Additionally the power supply did not have a stable output, which results in a varying magnetic field strength. 
Hence the error for the dispersion zone is set to be $\Delta D = \num{0.1}$. 
