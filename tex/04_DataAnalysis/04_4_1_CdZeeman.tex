\subsubsection{Zeeman Effect of Cadmium}
\label{toc:CdZeeman}
The current is set to $I = \SI{9}{\ampere}$ causing a magnetic field of $B = \SI{187.87 \pm 7.83}{\milli\tesla}$.
The following table shows the measured positions for polarisation \ang{0} and \ang{90}, now referring to $\sigma_1$ and $\sigma_2$. 
\begin{table}[ht]
	\centering
	\begin{minipage}{0.48\textwidth}
		\begin{adjustbox}{width=\textwidth}
			\centering
			\begin{tabular}{c c c c c}
				Ring Number		& $x_\text{left}$ [\si{\mm}]	& $x_\text{right}$ [\si{\mm}]	& $r$ [\si{\mm}]	& $r^2$ [\si{\mm\squared}]	\\
				\hline
				1			& \num{9.69 \pm 0.03} 		& \num{10.19 \pm 0.03} 		& \num{1.22 \pm 0.03} 	& \num{1.48 \pm 0.06} 		\\
				2			& \num{8.74 \pm 0.03} 		& \num{9.06 \pm 0.03} 		& \num{2.25 \pm 0.03} 	& \num{5.09 \pm 0.12} 		\\
				3			& \num{8.06 \pm 0.03} 		& \num{8.32 \pm 0.03} 		& \num{2.96 \pm 0.03} 	& \num{8.79 \pm 0.15} 		\\
				4			& \num{7.56 \pm 0.03} 		& \num{7.74 \pm 0.03} 		& \num{3.50 \pm 0.03} 	& \num{12.29 \pm 0.18} 		\\
				5			& \num{7.05 \pm 0.03} 		& \num{7.25 \pm 0.03} 		& \num{4.00 \pm 0.03} 	& \num{16.04 \pm 0.21} 		\\
%				6			& \num{6.67 \pm 0.03} 		& \num{6.86 \pm 0.03} 		& \num{4.39 \pm 0.03} 	& \num{19.27 \pm 0.23} 		\\ 
			\end{tabular}
		\end{adjustbox}
		\subcaption{Polarisation \ang{0} - $\sigma_1$}
	\end{minipage}
	\hfill
	\begin{minipage}{0.48\textwidth}
		\begin{adjustbox}{width=\textwidth}
			\centering
			\begin{tabular}{c c c c c}
				Ring Number		& $x_\text{left}$ [\si{\mm}]	& $x_\text{right}$ [\si{\mm}]	& $r$ [\si{\mm}]	& $r^2$ [\si{\mm\squared}]	\\
				\hline
				1			& \num{10.36 \pm 0.03} 		& \num{11.31 \pm 0.03} 		& \num{0.36 \pm 0.03} 	& \num{0.13 \pm 0.02} 		\\
				2			& \num{9.14 \pm 0.03} 		& \num{9.54 \pm 0.03} 		& \num{1.86 \pm 0.03} 	& \num{3.44 \pm 0.10} 		\\
				3			& \num{8.41 \pm 0.03} 		& \num{8.66 \pm 0.03} 		& \num{2.66 \pm 0.03} 	& \num{7.08 \pm 0.14} 		\\
				4			& \num{7.83 \pm 0.03} 		& \num{8.03 \pm 0.03} 		& \num{3.27 \pm 0.03} 	& \num{10.66 \pm 0.17} 		\\
				5			& \num{7.33 \pm 0.03} 		& \num{7.49 \pm 0.03} 		& \num{3.79 \pm 0.03} 	& \num{14.33 \pm 0.20} 		\\
%				6			& \num{6.87 \pm 0.03} 		& \num{6.99 \pm 0.03} 		& \num{4.27 \pm 0.03} 	& \num{18.19 \pm 0.22} 		\\
			\end{tabular}
		\end{adjustbox}
		\subcaption{Polarisation \ang{90} - $\sigma_2$}
	\end{minipage}
	\caption[Zeeman Splitting of Cd]{Zeeman splitting of cadmium (1/4 dispersion zone) for polarisation \ang{0} and \ang{90}}
\end{table}\\
The center position was determined to be $x_0 = \SI{11.18 \pm 0.01}{\mm}$. 
According to chapter~\ref{toc:WavelengthDetermination} the two wavelenghts are determined using a linar fit, which is shown in figure~\ref{fig:CdShift}.
\begin{figure}[ht]
	\centering
	\includegraphics[width=0.9\textwidth]{Images/CdPol.pdf}
	\caption[Determination of the Shifted Wavelength $\sigma_1$ \& $\sigma_2$ of Cd]{Determination of the shifted wavelength $\sigma_1$ and $\sigma_2$ of cadmium}
	\label{fig:CdShift}
\end{figure}\\
As a result one gets the relevant fit parameters and hence their corresponding wavelengths.
\begin{table}[ht]
	\centering
	\begin{tabular}{c p{2.0cm} c}
	$m_1 = \SI{3.61 \pm 0.03}{\mm\squared}$		& & $m_2 = \SI{3.59 \pm 0.04}{\mm\squared}$	\\
	$\lambda_{\sigma_1} = \SI{604.98 \pm 5.70}{\nm}$	& & $\lambda_{\sigma_2} = \SI{601.64 \pm 7.18}{\nm}$
	\end{tabular}
\end{table}\\
Now the $\sigma_1$~-~component can be assigned to $\sigma^-$, whereas $\sigma_2$ belongs to the $\sigma^+$~-~component. 
Since the $\sigma^-$~-~component has less energy than the $\sigma^+$~-~component, its radius turns out bigger. 
This would be consistent with the assignation, which should be correct, even though the two wavelength overlap within their interval of uncertainty. 
