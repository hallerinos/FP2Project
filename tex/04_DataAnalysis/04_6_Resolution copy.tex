\subsection{Resolution of the Fabry-P\'erot Interferometer}
\label{toc:ResolutionFPI}
The Full Width at Half Maximum $\delta\lambda$ is the wavelenght difference of two hardly distinguishable neighbouring maxima.
\begin{align}
	\delta\lambda = \Delta\lambda = \left|\lambda_{\text{max}_1}-\lambda_{\text{max}_2}\right|
\end{align}
During the observation of the Zeeman effect it was unfortunately neglected to memorise the current value for the splitting to start. 
However it is possible to estimate the resolution of the Fabry-P\'erot interferometer based on the available data, but the result is expected to be different from the actual value.
Therefore the wavelength difference of $\lambda_{\text{max}_1}-\lambda_{\text{max}_2}$ was chosen to be $\lambda_{\sigma_1}-\lambda_{\sigma_2}$.
The resolution estimates to
\begin{align}
	\left(\frac{\lambda}{\delta\lambda}\right) &= \frac{\lambda}{\left|\lambda_{\sigma_1}-\lambda_{\sigma_2}\right|}\\
	\left(\frac{\lambda}{\delta\lambda}\right) &= \frac{\lambda}{\left|\lambda_{\sigma_1}-\lambda_{\sigma_2}\right|}\cdot \sqrt{\left(\frac{\Delta\lambda}{\lambda}\right)^2 + \left(\frac{\Delta\lambda_{\sigma_1}}{\lambda_{\sigma_1}}\right)^2 + \left(\frac{\Delta\lambda_{\sigma_2}}{\lambda_{\sigma_2}}\right)^2}
\end{align}
Theoretically, the finesse $\mathcal{F}$ should be a reference value for the interferometer's resolution, if it is multiplied by $n_0$ (which was determined in chapter~\ref{toc:Wavelength}).
	\begin{align}
	\begin{split}
	\left(\frac{\lambda}{\delta\lambda}\right)_{\text{ref}} 	&= \frac{2 \mu d}{\lambda_\text{lit}} \cdot \frac{\pi \sqrt{R}}{1-R}\\
	\Delta \left(\frac{\lambda}{\delta\lambda}\right)_{\text{ref}}	&= \frac{2 \mu \Delta d}{\lambda_\text{lit}} \cdot \frac{\pi \sqrt{R}}{1-R}
	\end{split}
	\end{align}
\begin{table}[ht]
	\centering
	\begin{tabular}{c c c c}
	Element		& $\left(\frac{\lambda}{\delta\lambda}\right)_{\text{exp}}$	& $\left(\frac{\lambda}{\delta\lambda}\right)_{\text{ref}}$	& Deviation Order \\
	\hline
	Cd		& \num{186.329 \pm 3.1533}					& \num{146.619 \pm 0.971 e5}					& $\mathcal{O}(\num{e5})$	\\
	Sn		& \num{34.573 \pm 0.991}					& \num{14.838 \pm 0.098 e6}					& $\mathcal{O}(\num{e6})$	\\
	\end{tabular}
	\caption[Determination of the FPI's resolution]{Determination of the FPI's resolution using the zeeman splitting of cadmium and tin at $D\approx 1/4$}
	\label{tab:SpecCharge_e}
\end{table}\\
Therefore the deviation is of $\mathcal{O}(>10^5)$.
Possible origins of this are explained in more detail in chapter~\ref{toc:PlausibilityZeeman}.
