\subsubsection{Wavelength of Cadmium}
\label{toc:WavelengthCd}
For the singlet line of the cadmium lamp the center position of the interference pattern was determined to be $x_0 = \SI{11.20 \pm 0.01}{\mm}$, the values for each measurement are listed in the following table.
\begin{table}[ht]
	\centering
	\begin{tabular}{c c c}
	Ring Number		& $x_\text{left}$ [\si{\mm}]	& $x_\text{right}$ [\si{\mm}]	\\
	\hline
	1			& \num{9.78 \pm 0.03}	 	& \num{12.70 \pm 0.03}		 \\
	2			& \num{8.92 \pm 0.03} 		& \num{13.45 \pm 0.03}		 \\
	3			& \num{8.19 \pm 0.03} 		& \num{14.18 \pm 0.03} 		\\
	4			& \num{7.66 \pm 0.03} 		& \num{14.75 \pm 0.03} 		\\
	\end{tabular}
	\caption[Measurement - Center Position Cd]{Measured values to determine the center position for Cd}
	\label{tab:CdCenter}
\end{table}\\
Propagation of uncertainty yields a very small error in this case, which bases upon the averaging. 
A more realistic error for the center position can be obtained, if one calculates the standard derivation for the four individual, averaged positions. 
This results in\linebreak $x_0 = \SI{11.20 \pm 0.03}{\mm}$, which is used for further analysis. 
The resulting radii for the first to the tenth ring are listed in table~\ref{tab:CdRings}. 
\begin{table}[ht]
	\centering
	\begin{tabular}{c c c c c}
	Ring Number		& $x_\text{left}$ [\si{\mm}]	& $x_\text{right}$ [\si{\mm}]	& $r$ [\si{\mm}]		& $r^2$ [\si{\mm\squared}]	\\
	\hline
	1			& \num{9.69 \pm 0.03}		& \num{10.14 \pm 0.03}		& \num{1.29 \pm 0.02} 		& \num{1.66 \pm 0.06} \\
	2			& \num{8.77 \pm 0.03}		& \num{9.02 \pm 0.03}		& \num{2.31 \pm 0.02} 		& \num{5.33 \pm 0.11} \\
	3			& \num{8.09 \pm 0.03}		& \num{8.28 \pm 0.03}		& \num{3.02 \pm 0.02} 		& \num{9.11 \pm 0.14} \\
	4			& \num{7.59 \pm 0.03}		& \num{7.71 \pm 0.03}		& \num{3.55 \pm 0.02} 		& \num{12.63 \pm 0.17} \\
	5			& \num{7.06 \pm 0.03}		& \num{7.21 \pm 0.03}		& \num{4.07 \pm 0.02} 		& \num{16.55 \pm 0.19} \\
	6			& \num{6.68 \pm 0.03}		& \num{6.79 \pm 0.03}		& \num{4.47 \pm 0.02} 		& \num{19.97 \pm 0.21} \\
	7			& \num{6.25 \pm 0.03}		& \num{6.38 \pm 0.03}		& \num{4.89 \pm 0.02} 		& \num{23.90 \pm 0.23} \\
	8			& \num{5.87 \pm 0.03}		& \num{6.03 \pm 0.03}		& \num{5.25 \pm 0.02} 		& \num{27.60 \pm 0.25} \\
	9			& \num{5.56 \pm 0.03}		& \num{5.66 \pm 0.03}		& \num{5.59 \pm 0.02} 		& \num{31.29 \pm 0.27} \\
	10			& \num{5.19 \pm 0.03}		& \num{5.35 \pm 0.03}		& \num{5.93 \pm 0.02} 		& \num{35.21 \pm 0.28} \\
%	11			& \num{4.91 \pm 0.03}		& \num{5.04 \pm 0.03}		& \num{6.23 \pm 0.02} 		& \num{38.80 \pm 0.30} \\
	\end{tabular}
	\caption[Measurement - Ring Positions Cd]{Measured values to determine the radii for Cd}
	\label{tab:CdRings}
\end{table}\\
If the values for $r^2$ are plotted against the number of the rings $p$, one expects a linear relation of the form $r^2(p) = m\cdot p + b$, that is shown in figure~\ref{fig:CdWavelengthFit}.
The relevant parameter is determined to be $m = \SI{3.72 \pm 0.01}{\mm\squared}$, which yields a wavelength of
\begin{align}
	\lambda_\text{Cd,exp} = \SI{623.65 \pm 4.64}{\nm}
\end{align}
for the singlet cadmium line. 
A comparison to the literature value of $\lambda_\text{Cd,lit} = \SI{643.85}{\nm}$ shows a large discrepancy, which can have multiple sources.\\
The biggest deviation should originate from the Fabry-P\'erot interferometer, which is, as described in chapter~\ref{toc:FPIFinesse}, not as suitable to measure absolute wavelengths. 
Another source of errors is the fit, which yields a small error for parameter $m$ and a small $\chi^2$-value. 
This is traced back to the small relative errors of the data points, caused by the averaging.\\
\begin{figure}[ht]
	\centering
	\includegraphics[width=0.7\textwidth]{Images/zeeman2.pdf}
	\caption[Determination of the Wavelength of Cd]{Determination of the wavelength of Cd using a linear fit modell}
	\label{fig:CdWavelengthFit}
\end{figure}\\
Now that the wavelength is specified, it is possible to determine the order of the center ring, which is given by (refer to equation~\ref{eq:neq1})
%\begin{align}
%	n_0 = \frac{2d}{\lambda} \hspace{0.5cm} \text{with an error of} \hspace{0.5cm} \Delta n_0 = \n_0\frac{\left(\frac{\Delta d}{d}\right)^2 + \left(\frac{\Delta \lambda}{\lambda}\right)^2}\ .
%\end{align}
\begin{align}
	n_{0,\text{Cd}} = \num{48424.4 \pm 482.6}\ ,
\end{align}
with an error by propagation of uncertainty. 
In order to determine the order of maximum of the first ring $n_1 = \lbrack n_0 - \epsilon \rbrack$, one needs the interpolation of the linear fit to $r^2(p) = 0$, which yields $\epsilon = \num{0.57 \pm 0.02}$. 
The first ring's order is then determined to be
\begin{align}
	n_1 = \num{48424 \pm 483}\ .
\end{align}
Because of the large order of $n_0$ with a large uncertainty compared to $\epsilon$, the order of the first ring is specified by the order of $n_0$ with sufficient precision.
