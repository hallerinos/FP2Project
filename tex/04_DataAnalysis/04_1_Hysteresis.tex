\subsection{Measurement of the Magnetic Hysteresis}
\label{toc:Hysteresis}
In preparation of the Zeeman experiment one has to measure the hysteresis curve of the two coils in order to get a relation between the coil current and the magnetic flux density. 
A Hall probe was placed between the two coils at the position with the strongest field.  
It is important that it is adjusted vertically to the magnetic flux line to get the maximum density. 
Starting from \SI{0}{\ampere} the current was increased in steps of \SI{2}{\ampere} up to \SI{40}{\ampere}. 
Measuring the magnetic flux density at each step yields the hysteresis' upward curve. 
The downward curve was measured with the same step size starting from \SI{40}{\ampere} down to \SI{0}{\ampere}. 
Because of a non-stabilised power supply one has to estimate an error of \SI{0.5}{\ampere} for the coil current. 
The uncertainty of the flux density is set to be \SI{3}{\milli\tesla}, which includes possible systematic errors such as not measuring at the point of maximum field strength as well as statistic errors caused e.g. by errors in reading.\\
In order to determine both of the hysteresis curves a third degree polynomial fit is used. 
It is possible to use one of fourth degree, horever this did not have a considerable impact on the result. 
Figure~\ref{fig:MagHys} shows the data points with both of the fits. 
\begin{figure}[ht]
	\centering
	\includegraphics[width=0.85\textwidth]{Images/zeeman1.pdf}
	\caption[Magnetic Hysteresis Curve]{Polynomial fits of the magnetic upward and downward hysteresis curves}
	\label{fig:MagHys}
\end{figure}\\
It shows a small difference between the hysteresis' upward and downward curve, which results in a small are between those graphs. 
This area should not have a significant impact, why it may be neglected. 
Therefore the upward curve is used to transform the coil current into a magnetic flux density for further analysis.
