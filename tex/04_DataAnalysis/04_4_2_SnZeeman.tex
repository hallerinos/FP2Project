\subsubsection{Zeeman Effect of Tin}
\label{toc:SnZeeman}
In order to measure the shifted spectral lines for tin the current was set to $I = \SI{9.5}{\ampere}$, leading to a magnetic field of $B = \SI{197.81 \pm 8.04}{\milli\tesla}$. 
Again the two lines were seperated using the polarisation filter, the measured data is shown in table~\ref{tab:SnZeeman}.
\begin{table}[ht]
	\centering
	\begin{minipage}{0.48\textwidth}
		\begin{adjustbox}{width=\textwidth}
			\centering
			\begin{tabular}{c c c c c}
				Ring Number		& $x_\text{left}$ [\si{\mm}]	& $x_\text{right}$ [\si{\mm}]	& $r$ [\si{\mm}]	& $r^2$ [\si{\mm\squared}]	\\
				\hline
				1			& \num{10.69 \pm 0.03} 		& \num{11.23 \pm 0.03} 		& \num{0.23 \pm 0.02} 	& \num{0.05 \pm 0.01} 		\\
				2			& \num{9.33 \pm 0.03} 		& \num{9.66 \pm 0.03} 		& \num{1.69 \pm 0.02} 	& \num{2.87 \pm 0.08} 		\\
				3			& \num{8.54 \pm 0.03} 		& \num{8.78 \pm 0.03} 		& \num{2.53 \pm 0.02} 	& \num{6.39 \pm 0.12} 		\\
				4			& \num{7.95 \pm 0.03} 		& \num{8.09 \pm 0.03} 		& \num{3.17 \pm 0.02} 	& \num{10.04 \pm 0.15} 		\\
				5			& \num{7.43 \pm 0.03} 		& \num{7.57 \pm 0.03} 		& \num{3.69 \pm 0.02} 	& \num{13.61 \pm 0.17} 		\\
%				6			& \num{6.95 \pm 0.03} 		& \num{7.11 \pm 0.03} 		& \num{4.16 \pm 0.02} 	& \num{17.30 \pm 0.20} 		\\
			\end{tabular}
		\end{adjustbox}
		\subcaption{Polarisation \ang{0} - $\sigma_1$}
	\end{minipage}
	\hfill
	\begin{minipage}{0.48\textwidth}
		\begin{adjustbox}{width=\textwidth}
			\centering
			\begin{tabular}{c c c c c}
				Ring Number		& $x_\text{left}$ [\si{\mm}]	& $x_\text{right}$ [\si{\mm}]	& $r$ [\si{\mm}]	& $r^2$ [\si{\mm\squared}]	\\
				\hline
				1			& \num{9.89 \pm 0.03} 		& \num{10.50 \pm 0.03} 		& \num{0.99 \pm 0.02} 	& \num{0.99 \pm 0.05} 		\\
				2			& \num{8.91 \pm 0.03} 		& \num{9.20 \pm 0.03} 		& \num{2.13 \pm 0.02} 	& \num{4.55 \pm 0.10} 		\\
				3			& \num{8.16 \pm 0.03} 		& \num{8.50 \pm 0.03} 		& \num{2.86 \pm 0.02} 	& \num{8.17 \pm 0.14} 		\\
				4			& \num{7.70 \pm 0.03} 		& \num{7.88 \pm 0.03} 		& \num{3.40 \pm 0.02} 	& \num{11.55 \pm 0.16} 		\\
				5			& \num{7.23 \pm 0.03} 		& \num{7.36 \pm 0.03} 		& \num{3.89 \pm 0.02} 	& \num{15.16 \pm 0.18} 		\\
%				6			& \num{6.81 \pm 0.03} 		& \num{6.89 \pm 0.03} 		& \num{4.34 \pm 0.02} 	& \num{18.82 \pm 0.21} 		\\	
			\end{tabular}
		\end{adjustbox}
		\subcaption{Polarisation \ang{90} - $\sigma_2$}
	\end{minipage}
	\caption[Zeeman Splitting of Sn]{Zeeman splitting of tin (1/4 dispersion zone) for polarisation \ang{0} and \ang{90}}
	\label{tab:SnZeeman}
\end{table}\\
With a center position of $x_0 = \SI{11.19 \pm 0.01}{\mm}$ the wavelengths were determined via the fits, that are shown in figure~\ref{fig:SnShift}.
\begin{figure}[ht]
	\centering
	\includegraphics[width=0.9\textwidth]{Images/SnPol.pdf}
	\caption[Determination of the Shifted Wavelength $\sigma_1$ \& $\sigma_2$ of Sn]{Determination of the shifted wavelength $\sigma_1$ and $\sigma_2$ of tin}
	\label{fig:SnShift}
\end{figure}\\
As a result, the relevant parameters $m$ and their corresponding wavelengths are listed below.
\begin{table}[ht]
	\centering
	\begin{tabular}{c p{2.0cm} c}
		$m_1 = \SI{3.43 \pm 0.09}{\mm\squared}$		& & $m_2 = \SI{3.54 \pm 0.02}{\mm\squared}$	\\
		$\lambda_{\sigma_1} = \SI{575.16 \pm 1.54}{\nm}$	& & $\lambda_{\sigma_2} = \SI{593.03 \pm 3.51}{\nm}$
	\end{tabular}
\end{table}
In contrast to the measurement of cadmium, the assignation of the components is inverse. 
The $\sigma_1$~-~component can be associated with $\sigma^+$, whereas the $\sigma_2$~-~component belongs to $\sigma^-$, because it turns out smaller.
