\subsubsection{Hysteresis Loop}
The method used for fitting is clearly a wrong model for the real behaviour of a magnetic hysteresis loop. 
Usual magnetisation models use compositions of exponential functions of the form $\exp(-\beta H + \mu N)$, where $H$ is the systems Hamiltonian and $N$ the particlenumber operator.
This was tried to reproduce with arguments $\propto I^2$, but the result is not satisfying because of the huge parameter errors.
A set of measurements with smaller steps would probably yield a better result, the parameter errors are way to high to rate the plausibility of this model.
But since further analysis needs a propagation of the field's error, a polynomial fit is more suitable.
The good $\chi^2_{\text{red}}$ values for both loop branches support the satisfying approximation.
\begin{figure}[ht]
	\centering
	\includegraphics[width = 0.7\textwidth]{images/HysExp.pdf}
	\caption[Hysteresis Loop with a More Realistic Fit Model]{This model with compositions of exponential functions and arguments $\propto I^2$ should be more effective to fit the magnetising loop of a conductor, but since the error of $b$ and $e$ is huge, a more simply model has been used for the analysis.}
\end{figure}\\
Another error may be the delay between setting the current and the change of the magnetic field caused by self inductance.
The impact of this effect can not be easily estimated, because the power supply is not stable enough.
