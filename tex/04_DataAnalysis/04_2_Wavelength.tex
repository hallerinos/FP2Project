\subsection{Determination of the Wavelength}
\label{toc:Wavelength}
For this task the magnetic field stays off. 
With help of the camera one can observe the interference pattern on the computer, the camera can be adjourned in order to measure distances between the rings. 
Figure~\ref{fig:InterferenceRings} shows the rings produced by the Fabry-P\'erot interferometer.
\begin{figure}[ht]	 
	\centering
	\includegraphics[width=0.5\textwidth]{Images/InterferenceRings.jpg}
	\caption[Interference Rings of the Fabry-P\'erot Interferometer]{Interference rings of the Fabry-P\'erot interferometer. The small needle is used to mark the position and measure distances.}
	\label{fig:InterferenceRings}
\end{figure}\\
On the monitor one observes the unshifted interference rings for cadmium and for tin ($\lambda_\text{Cd,lit} = \SI{643.85}{\nano\metre}$, $\lambda_\text{Sn,lit} = \SI{636.23}{\nm}$) over ten orders. 
To improve the accuracy of the measurement, the position for each ring is determined by calculating the mean value of the left and the right end according to
\begin{align}
	x_i = \frac{x_{i,\text{left}} + x_{i,\text{right}}}{2} \hspace{0.5cm} \text{with} \hspace{0.5cm} \Delta x_i = \frac{1}{2}\sqrt{\left(\Delta x_{i,\text{left}}\right)^2 + \left(\Delta x_{i,\text{right}}\right)^2}\ .
\end{align}
This approach should deliver a more precise result for the center of the rings, since they are of different thickness. 
The uncertainty in each measured value is assumed to be $\Delta x = \SI{0.03}{\mm}$, even though the resolution of the scale is \SI{0.01}{\mm}. This is based on the fact, that the rings constrict for lower orders and therefore are more difficult to measure exactly.\\
To calculate the radius of each ring one needs to know the position of the center $x_0$. 
This can be achieved by measuring the first ring's position $x'$ left and right to the middle and taking the mean.
\begin{align}
	\begin{split}
		x_0 &= \frac{x'_{1,\text{left}} + x'_{1,\text{right}}}{2}\\
		\Delta x_0 &= \frac{1}{2}\sqrt{(\Delta x'_{1,\text{left}})^2 + (\Delta x'_{1,\text{right}})^2}\ .
	\end{split}
\end{align}  
To improve the measurement not only the first ring, but the first four rings were used and averaged afterwards.
The rings' radii now calculate according to
\begin{align}
	r_i = x_0 - x_i \hspace{0.5cm} \text{with an error of} \hspace{0.5cm} \Delta r_i = \sqrt{(\Delta x_0)^2 + (\Delta x_i)^2}\ .
\end{align}
Figure~\ref{fig:InterferenceRings} only shows the left side of the interference pattern. 
The right side could be made visible, if the optical elements in the course of beam were adjusted a little bit.
\subsubsection{Wavelength of Cadmium}
\label{toc:WavelengthCd}
For the singlet line of the cadmium lamp the center position of the interference pattern was determined to be $x_0 = \SI{11.20 \pm 0.01}{\mm}$, the values for each measurement are listed in the following table.
\begin{table}[ht]
	\centering
	\begin{tabular}{c c c}
	Ring Number		& $x_\text{left}$ [\si{\mm}]	& $x_\text{right}$ [\si{\mm}]	\\
	\hline
	1			& \num{9.78 \pm 0.03}	 	& \num{12.70 \pm 0.03}		 \\
	2			& \num{8.92 \pm 0.03} 		& \num{13.45 \pm 0.03}		 \\
	3			& \num{8.19 \pm 0.03} 		& \num{14.18 \pm 0.03} 		\\
	4			& \num{7.66 \pm 0.03} 		& \num{14.75 \pm 0.03} 		\\
	\end{tabular}
	\caption[Measurement - Center Position Cd]{Measured values to determine the center position for Cd}
	\label{tab:CdCenter}
\end{table}\\
Propagation of uncertainty yields a very small error in this case, which bases upon the averaging. 
A more realistic error for the center position can be obtained, if one calculates the standard derivation for the four individual, averaged positions. 
This results in\linebreak $x_0 = \SI{11.20 \pm 0.03}{\mm}$, which is used for further analysis. 
The resulting radii for the first to the tenth ring are listed in table~\ref{tab:CdRings}. 
\begin{table}[ht]
	\centering
	\begin{tabular}{c c c c c}
	Ring Number		& $x_\text{left}$ [\si{\mm}]	& $x_\text{right}$ [\si{\mm}]	& $r$ [\si{\mm}]		& $r^2$ [\si{\mm\squared}]	\\
	\hline
	1			& \num{9.69 \pm 0.03}		& \num{10.14 \pm 0.03}		& \num{1.29 \pm 0.02} 		& \num{1.66 \pm 0.06} \\
	2			& \num{8.77 \pm 0.03}		& \num{9.02 \pm 0.03}		& \num{2.31 \pm 0.02} 		& \num{5.33 \pm 0.11} \\
	3			& \num{8.09 \pm 0.03}		& \num{8.28 \pm 0.03}		& \num{3.02 \pm 0.02} 		& \num{9.11 \pm 0.14} \\
	4			& \num{7.59 \pm 0.03}		& \num{7.71 \pm 0.03}		& \num{3.55 \pm 0.02} 		& \num{12.63 \pm 0.17} \\
	5			& \num{7.06 \pm 0.03}		& \num{7.21 \pm 0.03}		& \num{4.07 \pm 0.02} 		& \num{16.55 \pm 0.19} \\
	6			& \num{6.68 \pm 0.03}		& \num{6.79 \pm 0.03}		& \num{4.47 \pm 0.02} 		& \num{19.97 \pm 0.21} \\
	7			& \num{6.25 \pm 0.03}		& \num{6.38 \pm 0.03}		& \num{4.89 \pm 0.02} 		& \num{23.90 \pm 0.23} \\
	8			& \num{5.87 \pm 0.03}		& \num{6.03 \pm 0.03}		& \num{5.25 \pm 0.02} 		& \num{27.60 \pm 0.25} \\
	9			& \num{5.56 \pm 0.03}		& \num{5.66 \pm 0.03}		& \num{5.59 \pm 0.02} 		& \num{31.29 \pm 0.27} \\
	10			& \num{5.19 \pm 0.03}		& \num{5.35 \pm 0.03}		& \num{5.93 \pm 0.02} 		& \num{35.21 \pm 0.28} \\
%	11			& \num{4.91 \pm 0.03}		& \num{5.04 \pm 0.03}		& \num{6.23 \pm 0.02} 		& \num{38.80 \pm 0.30} \\
	\end{tabular}
	\caption[Measurement - Ring Positions Cd]{Measured values to determine the radii for Cd}
	\label{tab:CdRings}
\end{table}\\
If the values for $r^2$ are plotted against the number of the rings $p$, one expects a linear relation of the form $r^2(p) = m\cdot p + b$, that is shown in figure~\ref{fig:CdWavelengthFit}.
The relevant parameter is determined to be $m = \SI{3.72 \pm 0.01}{\mm\squared}$, which yields a wavelength of
\begin{align}
	\lambda_\text{Cd,exp} = \SI{623.65 \pm 4.64}{\nm}
\end{align}
for the singlet cadmium line. 
A comparison to the literature value of $\lambda_\text{Cd,lit} = \SI{643.85}{\nm}$ shows a large discrepancy, which can have multiple sources.\\
The biggest deviation should originate from the Fabry-P\'erot interferometer, which is, as described in chapter~\ref{toc:FPIFinesse}, not as suitable to measure absolute wavelengths. 
Another source of errors is the fit, which yields a small error for parameter $m$ and a small $\chi^2$-value. 
This is traced back to the small relative errors of the data points, caused by the averaging.\\
\begin{figure}[ht]
	\centering
	\includegraphics[width=0.7\textwidth]{Images/zeeman2.pdf}
	\caption[Determination of the Wavelength of Cd]{Determination of the wavelength of Cd using a linear fit modell}
	\label{fig:CdWavelengthFit}
\end{figure}\\
Now that the wavelength is specified, it is possible to determine the order of the center ring, which is given by (refer to equation~\ref{eq:neq1})
%\begin{align}
%	n_0 = \frac{2d}{\lambda} \hspace{0.5cm} \text{with an error of} \hspace{0.5cm} \Delta n_0 = \n_0\frac{\left(\frac{\Delta d}{d}\right)^2 + \left(\frac{\Delta \lambda}{\lambda}\right)^2}\ .
%\end{align}
\begin{align}
	n_{0,\text{Cd}} = \num{48424.4 \pm 482.6}\ ,
\end{align}
with an error by propagation of uncertainty. 
In order to determine the order of maximum of the first ring $n_1 = \lbrack n_0 - \epsilon \rbrack$, one needs the interpolation of the linear fit to $r^2(p) = 0$, which yields $\epsilon = \num{0.57 \pm 0.02}$. 
The first ring's order is then determined to be
\begin{align}
	n_1 = \num{48424 \pm 483}\ .
\end{align}
Because of the large order of $n_0$ with a large uncertainty compared to $\epsilon$, the order of the first ring is specified by the order of $n_0$ with sufficient precision.

\subsubsection{Wavelength of Tin}
\label{toc:WavelengthSn}
For the singlet line of the tin lamp the center position of the interference pattern was determined to be $x_0 = \SI{11.20 \pm 0.01}{\mm}$, the values for each measurement are listed in the following table.
\begin{table}[ht]
	\centering
	\begin{tabular}{c c c}
	Ring Number		& $x_\text{left}$ [\si{\mm}]	& $x_\text{right}$ [\si{\mm}]	\\
	\hline
	1			& \num{10.28 \pm 0.03} 		& \num{12.13 \pm 0.03}		 \\
	2			& \num{9.07 \pm 0.03} 		& \num{13.31 \pm 0.03}		 \\
	3			& \num{8.33 \pm 0.03} 		& \num{14.06 \pm 0.03}		 \\
	4			& \num{7.74 \pm 0.03} 		& \num{14.65 \pm 0.03}		 \\
	\end{tabular}
	\caption[Measurement - Center Position Sn]{Measured values to determine the center position for Sn}
	\label{tab:SnCenter}
\end{table}\\
Propagation of uncertainty again yields a small error for the center position. 
The methode of calculation the standard derivation leads to an even smaller error, which shows a precise measurement. 
The values used for the linear fit are listed in table~\ref{tab:SnRings}, the fit itself is shown in figure~\ref{fig:SnWavelengthFit}.
\begin{table}[ht]
	\centering
	\begin{tabular}{c c c c c}
	Ring Number		& $x_\text{left}$ [\si{\mm}]	& $x_\text{right}$ [\si{\mm}]	& $r$ [\si{\mm}]		& $r^2$ [\si{\mm\squared}]	\\
	\hline
	1			& \num{9.97 \pm 0.03} 		& \num{10.37 \pm 0.03} 		& \num{1.03 \pm 0.02} 		& \num{1.05 \pm 0.05} \\
	2			& \num{8.92 \pm 0.03} 		& \num{9.19 \pm 0.03} 		& \num{2.14 \pm 0.02} 		& \num{4.58 \pm 0.10} \\
	3			& \num{8.24 \pm 0.03} 		& \num{8.43 \pm 0.03} 		& \num{2.86 \pm 0.02} 		& \num{8.19 \pm 0.14} \\
	4			& \num{7.68 \pm 0.03} 		& \num{7.84 \pm 0.03} 		& \num{3.44 \pm 0.02} 		& \num{11.81 \pm 0.16} \\
	5			& \num{7.17 \pm 0.03} 		& \num{7.30 \pm 0.03} 		& \num{3.96 \pm 0.02} 		& \num{15.69 \pm 0.19} \\
	6			& \num{6.72 \pm 0.03} 		& \num{6.86 \pm 0.03} 		& \num{4.41 \pm 0.02} 		& \num{19.42 \pm 0.21} \\
	7			& \num{6.31 \pm 0.03} 		& \num{6.47 \pm 0.03} 		& \num{4.81 \pm 0.02} 		& \num{23.10 \pm 0.23} \\
	8			& \num{5.89 \pm 0.03} 		& \num{6.10 \pm 0.03} 		& \num{5.20 \pm 0.02} 		& \num{27.05 \pm 0.25} \\
	9			& \num{5.61 \pm 0.03} 		& \num{5.76 \pm 0.03} 		& \num{5.51 \pm 0.02} 		& \num{30.37 \pm 0.26} \\
	10			& \num{5.31 \pm 0.03} 		& \num{5.45 \pm 0.03} 		& \num{5.82 \pm 0.02} 		& \num{33.83 \pm 0.28} \\
%	11			& \num{5.02 \pm 0.03} 		& \num{5.17 \pm 0.03} 		& \num{6.10 \pm 0.02} 		& \num{37.23 \pm 0.29} \\
	\end{tabular}
	\caption[Measurement - Ring Positions Sn]{Measured values to determine the radii for Sn}
	\label{tab:SnRings}
\end{table}\\
The relevant fit parameter is determined to be $m = \SI{3.68 \pm 0.02}{\mm\squared}$, which results in an experimental wavelength of
\begin{align}
	\lambda_\text{Sn,exp} = \SI{617.66 \pm 5.22}{\nm}
\end{align}
for the singlet tin line. 
Compared to the literature value of $\lambda_\text{Sn,lit} = \SI{636.23}{\nm}$, the measurement shows a better conformity than for the singlet line of cadmium. 
Still the experimental value does not include the literature value in its interval. 
The reasons are the same as for the determination of the cadmium wavelength.
\begin{figure}[ht]
	\centering
	\includegraphics[width=0.7\textwidth]{Images/zeeman3.pdf}
	\caption[Determination of the Wavelength of Sn]{Determination of the wavelength of Sn using a linear fit modell}
	\label{fig:SnWavelengthFit}
\end{figure}\\
With the specified wavelength the order of the center ring can be determined to be
\begin{align}
	n_0 = \num{48894.2 \pm 525.0}\ ,
\end{align}
which yields the order of the first maximum using $\epsilon = \num{0.74 \pm 0.03}$
\begin{align}
	n_1 = \num{48893 \pm 525}\ .
\end{align}

