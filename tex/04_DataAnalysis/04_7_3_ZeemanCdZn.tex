\subsubsection{Bohr Magneton, Specific Charge and Resolution}
\label{toc:PlausibilityZeeman}

The result of the determined values for $\lambda_{\sigma^\pm}$ is calculated with the same uncertainty as the wavelengths of the previous chapter~\ref{toc:PlausibilityWavelength}.
However, this does not matter for the wavelength differences and the calculated value $\overline{\delta\nu}$. 
With the measurements of tin and cadmium it is allowed to use the mean value to calculate a reference for the bohr magneton $\mu_\text{B}$. 
The relative error $\frac{\Delta\mu_\text{B}}{\mu_\text{B}} = \SI{17.67}{\percent}$ arises of the small difference of the denominator in equation~\ref{eq:NuOrder}. 
A smaller error would occur with a bigger lense, which results in a better seperation of the interference rings, therefore a bigger value of the mentioned denominator and a smaller relative error of $r_p$.
The FPI's resolution shows a deviation of $\mathcal{O}(10^2)$ to the theoretical value.
This can not be explained with the mistaken magnetic field at the first noticeable splitting, because at $B \approx \mathcal{O}(\SI{1}{\milli\tesla})$ clearly no shift can be remarked.
Particularly eyeballing the intensity distribution is a huge mistake for the start of the line splitting.
Although this errors may be considered, they can not be the only reason for this huge deviation.
More likely, the assumed reflection coefficient $R$ is not the real value, considering the mirrors may be fogged in course of time.
