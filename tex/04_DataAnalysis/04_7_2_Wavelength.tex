\subsubsection{Determination of the Wavelength}
\label{toc:PlausibilityWavelength}
The wavelenght determination with help of the squared equation~\ref{eq:RSqLambda} is disaffecting, because the wavelength's relative deviation is beyond the error intervals.
Systematical errors of the setup's scaling can be excluded, considering the fit's slope does not depend on any offsets.
A proper explanation may be the setting of the plane at the focal distance. 
If this plane is not placed vertically to the optical axis, the distance of the outer rings to the focal lense differs significantly from the given distance $f$.
One can see this effect in figure~\ref{fig:InterferenceRings}, because the right side of the rings is more defocussed.
A big problem of this effect is its sensibility on tremor.
Between the singlet lines and the $\sigma^{\pm}$ measurement the lamp has been changed and the setup experienced some tremor, hence $\left[\lambda_{\sigma^+}, \lambda_{\sigma^-}\right]$ does not include the singlet wavelengths $\lambda_{\text{exp}}$.
