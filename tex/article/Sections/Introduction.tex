\section{Introduction}
Many body problems often cannot be resolved analytically because of their huge amount of accessible states, for an instance.
In order to solve such a problem with computational assistance, a variety of methods have been developed and all of them have their very own applications and limits.
One of the first models is the Monte Carlo simulation with the Metropolis Criterion.
Its nature is pure stochastic - the time progression evolves randomly and is not given a set of initial conditions.
% The timeframe of its creation is the 1950's during the Manhatten Project by Stanislav Ulam\footnote{$^*$April 13, 1909, $^\dag$May 13, 1984. Ulam got the idea of the MC approach indirectly while he was thinking about the probability of the popular game Solitaire's successful outcome.} and Jan Von Neumann\footnote{$^*$December 28, 1903, $^\dag$February 8, 1957.}.
However, the Molecular Dynamics approach is used for the same initial System - with equal coordinate initialisation and boundary conditions - by solving Newton's equations of motions to all atoms simultaniously.
This implies that this kind of simulations are deterministic and can be used to look into a system's time evolution.\medskip\\
The main goal of this work is an implementation of both methods in C++ for a set of noninteracting particles in a finite box with periodic boundary conditions and Lennard-Jones Potential.
In order to deepen the similarities and differences in computational and physical aspects, a comparison at the very end is suffitient.
