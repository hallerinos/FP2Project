\section{System Considerations}
If not explicitly mentioned, the system's settings are as follows:
\subsubsection*{Ensemble}
$N$ particles in a box of volume $V$ and temperature $T$ are considered.
The closed box is placed in an external heat bath, hence the total energy is not fixed and the probability $P_i$ for a given State $\ket{i}$ with Energy $E_i$ and $\beta \mathrel{\mathop:}= \frac{1}{kT}$ is given by
\begin{align}
	P_i = \frac{1}{Z}\exp\left(-\beta E_i\right)\text{, and }
	Z = \sum_k \exp\left(-\beta E_k\right).
\end{align}
Such an ensemble is called canonical or NVT.

\subsubsection*{Potential and Energy}
The particles are interacting via a normed Lennard-Jones Potential
\begin{align}
	V_{ij} = 4\left(\frac{1}{r_{ij}^{12}} - \frac{1}{r_{ij}^6} + \frac{2^7 - 1}{2^{14}}\right),
\end{align}
such that $V_{ij}=0$, if $r_{ij} = 2\sqrt[6]2$.
The total energy $E_n$ of the system in a state $\ket{n}$ is given by the expression
\begin{align}
	E_n = \sum_i\sum_{j\neq i}V_{ij}.
\end{align}

\subsubsection*{Boundary Conditions}
For both implementations, periodic boundary conditions (PBC) will be used.
More detailed - the particles at the border of the box can interact with fake neighbour particles beyond the scope of the border.
To understand the principle of this condition, a simple example is sufficient: Consider a one dimensional chain with next-neighbour interacting particles at fixed lattice points.
Each one has exactly two neighbours, except for the two at the border.
This incident can be terminated when telling the border particles to interact with one another.
This yields a positive effect on the system's scale parameters - box length $L$ and number of particles $N$ - they are implicitely bigger without the need for significantly more calculation steps.
