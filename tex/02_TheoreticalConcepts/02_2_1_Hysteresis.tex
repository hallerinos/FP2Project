\subsubsection{Magnetic Hysteresis}
\label{toc:Hysteresis}
Magnetic hysteresis describes the correlation between the magnetic field $H$ and the magnetic flux density $B$, which in in general non-linear. 
An example can be seen in figure~\ref{fig:MagHysteresis}.
\begin{figure}[h]
	\centering
	\includegraphics[width=0.7\textwidth]{Images/Hysteresis.pdf}
	\caption[Magnetic Hysteresis]{Correlation between the magnetic field $H$ and the flux density $B$ \cite{Hysteresis}}
	\label{fig:MagHysteresis}
\end{figure}\\
It is notable, that there are two different gradients for $B(H)$, dependent on whether the magnetic field strength $H$ is increased or decreased. 
For both very high or very low values there is a saturation value, which marks the turning point. 
For the sake of comparability it is inevitable to know which gradient is used during a measurement.

