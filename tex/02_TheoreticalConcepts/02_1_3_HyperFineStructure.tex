\subsubsection{Hyperfine Structure}
\label{toc:HyperFineStructure}

Besides interaction between electons in the atom's shell, interactions between electrons and higher nucleus moments (e.g. dipole moment, quadrupole moment) are included in the hyperfine structure. 
It gives a more detailed shift of the spectral lines in the atomic spectrum and includes the coupling of the total electron angular momentum $\vec{J}$ and the nuclear spin $\vec{I}$. 
Starting with the magnetic field 
	\begin{align}
		\vec{B}_{J} = -B_0 \frac{\vec{J}}{|\vec{J}|}
	\end{align}
generated by a shell electron, which interacts with the magnetic momentum of the nucleus
	\begin{align}
		\vec{\mu}_{\text{N}} = g_{I}\mu_{\text{N}}\frac{\vec{I}}{\hbar}\ ,
	\end{align}
the Hamiltonian hyperfine structure term calculates to
	\begin{align}
		H_{\text{HFS}} = -\vec{\mu}_{\text{N}}\vec{B}_{J} = \frac{g_{I}\mu_{\text{N}}B_0}{\hbar^2\sqrt{j(j+1)}}\vec{I}\vec{J}\ .
	\end{align}
$\mu_{\text{N}} = \frac{e\hbar}{2m_{\text{p}}}$ refers to the nuclear magneton with the proton mass $m_{\text{p}}$ and $g_{I}$ is the nuclear g-factor. In order to get the eigenvalue of $\vec{I}\vec{J}$ one introduces the total angular momentum $\vec{F} = \vec{I} + \vec{J}$ of the shell and the core. 
This yields an additional energy correction of
	\begin{align}
		 \Delta E_{\text{HFS}} = \frac{g_{I}\mu_{\text{N}}B_0}{\hbar^2\sqrt{j(j+1)}}(f(f+1)-i(i+1)-j(i+1))\ ,
	\end{align}
which is significantly smaller than the FS correction.
