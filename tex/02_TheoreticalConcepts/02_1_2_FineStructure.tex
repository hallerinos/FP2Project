\subsubsection{Fine Structure}
\label{toc:FineStructure}
%	Every energy level can be assigned with more than one electron, which leads to degeneracy. 
%	The fine structure describes the splitting of these degenerated energy levels due to internal interaction and includes corrections for both relativistic effects and spin-orbit coupling. 
%	This leads to a Hamiltonian
%		\begin{align*}
%			H_{\text{FS}} = H_{\text{0}} + H_{\text{R}} + H_{\text{LS}}\ .
%		\end{align*}   
%	With methods of the perturbation theory and the assumption, that the spin-orbit coupling term is only a minor disturbance, it can be tranformed to be
%		\begin{align}
%			H_{\text{LS}} = \xi(r)\vec{L}\vec{S}\ ,
%		\end{align}
%	where $\xi(r) = \frac{\hbar}{c}\frac{\alpha}{2m_e^2r^3}$ is the coupling constant and $\alpha \approx \frac{1}{137}$. 
%	LS coupling is relevant for atoms with small atomic numbers $Z$. 
%	It describes the coupling of all electron spins to a spin momentum S and all electron angular momentums to a angular momentum L. 
%	Since the Hamiltonian does not commute with $\vec{L}$ or $\vec{S}$, but only with $\vec{L}^2$ and $\vec{S}^2$, a new quantum number is needed.\\
%	One has to find the eigenvalues for $\vec{L}\vec{S}$ which is possible with the total angular momentum
%		\begin{align*}
%			\vec{J} = \vec{L} + \vec{S}\ .
%		\end{align*}
%	The transformation of $\vec{J}^2 = (\vec{L} + \vec{S})(\vec{L} + \vec{S})$ leads to $\vec{L}\vec{S} = \frac{1}{2}(\vec{J}^2 - \vec{L}^2 - \vec{S}^2)$, so that the eigenvalue is given by
%		\begin{align}
%			\langle \vec{L}\vec{S} \rangle = \frac{\hbar^2}{2}(j(j+1) - l(l+1) - s(s+1))\ .
%		\end{align}
%	In order to determine the energy correction due to LS coupling one needs to calculate $\langle \frac{1}{r^3}\rangle$ using the Feynman-Hellman-Theorem
%		\begin{align}
%			\langle r^{-3} \rangle = \frac{1}{l(l+\frac{1}{2})(l+1)}\frac{1}{n^3a^3}\ ,
%		\end{align}
%	with $a = \frac{4\pi\epsilon_0\hbar^2}{m_e c^2}$ beeing the Bohr radius and the quantum number $n$. 
%	Finally the energy correction is given by
%		\begin{align}
%			\Delta E_{\text{LS}} = \frac{\alpha^2 m_e c^2}{4} \cdot n \cdot \frac{j(j+1)-l(l+1)-s(s+1)}{l(l+\frac{1}{2})(l+1)}\ . 
%		\end{align} 
%	Every state is still $(2j+1)$-times degenerated.
To understand the principles of fine and hyperfine structure splitting it is sufficient to review the energy lines for hydrogen.
Assuming the Bohr theory as correct, the energy levels are constituted by
	\begin{align}
	\label{eq:EBohr}
	E_n 	= \frac{E_1}{n^2} 
	\hspace{1.5cm} \text{with} \hspace{1.5cm} 
	E_1 	= - \frac{m_e}{2 \hbar^2}\left(\frac{e^2}{4 \pi \epsilon_0}\right)^2 
		= - \frac{\alpha^2}{2}\cdot m_e c^2\ ,
	\end{align}
where $\alpha = \frac{e^2}{4 \pi \epsilon_0}$ is the fine structure constant. 
It denotes the strength of electromagnetic interaction in common.
To correct the energy levels of eq.~\ref{eq:EBohr}, one has to consider the following effects.
	
	\begin{enumerate}
	\item{
	\begin{bf}Relativistic Corrections\end{bf}\\
	The non-relativistic kinetic energy is given by 
	\begin{align*}
	\Braket{T} = \braket{\frac{p^2}{2m}}\ .
	\end{align*} 
	To apply a relativistic correction, a taylor series expansion of the relativistic energy-impulse equation to the order $\mathcal{O}(p^6)$ leads to
	\begin{align*}
	E 	&= mc^2\cdot\sqrt{\left(\frac{pc}{mc^2}\right)^2+1}\\
		&\approx \frac{p^2}{2m}\left(1 - \frac{p^2}{4 m^2c^2} \right)\ . 
	\end{align*}
	The result of this is a hamiltonian consisting of an undisturbed term $H^0=\frac{\hat{p}^2}{{2m}} - \frac{\alpha\hbar c}{r}$ with the undisturbed hydrogen wavefunction $\Ket{\psi^0}$ and a small \linebreak
	pertubation $H'_\text{rel} = - \frac{\hat{p}^4}{8m^3c^2}$.
	This relativistic corrections may be done except for non-relativistic fermions, because 
	\begin{align*}
	\frac{\Braket{T}}{mc^2} \approx \frac{\mathcal{O}(10^1\si{\electronvolt})}{\mathcal{O}(10^2\si{\kilo\electronvolt})} = \mathcal{O}(10^{-4})\ .
	\end{align*}
	}
	
	\item{
	\begin{bf}Spin - Orbit Coupling\end{bf}\\
	Another pertubation term is the spin-orbit hamiltonian $H'_{\text{so}}=\frac{\alpha \hbar}{2 m^2r^3} \vec{L}\cdot\vec{S}$. 
	Its terms can be understood by classical calculations for a relation $\vec{B}(\vec{L},r) = \frac{\alpha}{e}\cdot \frac{\vec{L}}{m r^3}$.
	The hamiltonian $H'_{\text{so}}=-\vec{\mu}\cdot\vec{B}$ with the magnetic moment $\vec{\mu}=\gamma\cdot\vec{S}$, where $\gamma=\frac{-e}{m}$ for electrons, leads to the equation.
	}		
	\end{enumerate}
Further steps using time independent first order pertubation theory
\begin{align}
	E'_{\text{rel}} = \Braket{\psi^0 | H'_{\text{rel}} | \psi^0}
\end{align}
result in the correction terms for relativistic and spin-orbit considerations
	\begin{align}
	E'_{\text{rel},n} 	&= - \frac{E_n^2}{2 m c^2}\left( -3 + \frac{4n}{l + 1/2}\right)\\
	E'_{\text{so},n,j,l}	&= \frac{n E_n^2}{mc^2}\cdot\frac{j(j+1) - l(l+1) - 3/4}{l(l+1/2)(l+1)}\ .
	\end{align}
Both correction terms are very small compared to $E_n$ because of the supression $\mathcal{O}(\alpha^4)$.
Simplifying the sum of both terms, the fine structure correction is denoted to
	\begin{align}
	E_{\text{FS}} = \frac{E_n^2}{2mc^2}\left(3-\frac{4n}{j+1/2}\right) \ .
	\end{align}
