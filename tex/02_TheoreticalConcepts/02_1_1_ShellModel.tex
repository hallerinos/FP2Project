\subsubsection{Shell Model and Atomic Spectrum}
\label{toc:ShellModel}
In a simple model an atom is described as a nucleus with surrounding electons in a shell. 
This shell consists of different orbitals, on which an electron can remain. 
Each orbital represents a different energy level and can be characterised by the principal quantum number $n$. 
To get to an orbital with a higher energy, the electron needs to absorb a photon with a specific frequency. 
When changing to an orbital with a lower energy, the electron emitts a photon with a specific frequency.
The frequency depends on the transition energy between the two orbitals and is hence given by
	\begin{align}
		\nu = \frac{E_2-E_1}{h}\ .
	\end{align}
This forms an unique atomic spectrum, since only characteristic and discrete photon energies can be absorbed and emitted.
