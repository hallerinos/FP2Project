\subsubsection{Finesse of the Fabry-P\'erot Interferometer}
\label{toc:FPIFinesse}
The intensity distribution of a Fabry-P\'erot interferometer is given by the Airy-Function
\begin{align}
	\begin{split}
	I(\theta,\nu)	 	&=\frac{I_0}{1 + F\sin^2(\pi\frac{2d\cos(\theta)}{\lambda})}	\hspace{1.0cm}\text{with } F=\frac{4R}{(1-R)^2}\\
				&=\frac{I_0}{1 + F\sin^2(\pi\nu\cdot2d\cos(\theta))}		\hspace{1.0cm}\text{with } \nu=\frac{1}{\lambda}\ .
	\end{split}
\end{align}
To resolve it to the setup's capacity, one has to evaluate an expression for the Full Width at Half Maximum $\delta \nu$.
\begin{align}
	\begin{split}
	\frac{I(\theta,0)}{2} 	&= I(\theta, \frac{\delta \nu}{2}) \\
	\frac{1}{2}		&= \frac{1}{1 + F \sin^2(\pi \frac{\delta \nu}{2} \cdot 2d \cos(\theta))}\\
				&\mathrel{\makebox[\widthof{=}]{\vdots}} \\
	\frac{1}{\delta \nu}	&= \sqrt{F}\cdot\pi d
	\end{split}
\end{align}
This expression can be resolved to the setup's finesse $\mathcal{F}$
\begin{align}
	\label{eq:FPIFinesse}
	\mathcal{F} = \frac{\nu}{\delta\nu} = \frac{\pi\sqrt{F}}{2} = \frac{\pi \sqrt{R}}{1-R}\ ,
\end{align}
which is the usual quality reference value of the resolution.
The setup's Fabry-P\'erot interferometer has a reflection coefficient of $R=\num{0.99}$, which is a completely ideal and not measured value.
Therefore it is difficult to reckon the finesse's error.
The distance of the reflecting planes is denoted to $d=\SI{1.51\pm 0.01}{\centi\meter}$.\\
\\
\\
\textbf{Explanatory Note}\\
\\
It is important to know, that the Fabry-P\'erot interferometer used in this experiment is not as suitable to measure absolute wavelengths. 
To do so, it would be neccessary to insert a prism into the course of beam in order so separate the wavelengths first. 
This would cause a smaller spectrum of wavelengths, so that the interference pattern would be more suitable to measure the separated wavelength precisely.\\
However this experimental setup is qualified to measure small wavelength differences, hence it can be used to precisely measure wavelength shifts.
