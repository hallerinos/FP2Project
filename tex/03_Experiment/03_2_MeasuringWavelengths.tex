\subsection{Determination of Wavelengths}
\label{toc:WavelengthDetermination}
The wavelength of the cadmium and tin lamp is calculatable with the radii of the inner interference rings up to the tenth order.
It is measured with the experiment's setup without polarisation filters.
An adjustable needle with fixed camera is placed at the focal plane of the FPI to read out its relative position.
To calibrate the scaling, it is important to determine the spectrum's center position.
If one wants to measure the wavelength difference $\Delta \lambda$ and to compare it to a theoretical model, it is sufficient to use the relation resulting from following calculations.
Starting from equation~\ref{eq:ConstructiveInterferenceFabry}, one can write this expression as
\begin{align}
	\label{eq:neq1}
	\begin{split}
	n = n_0 \cdot (1 - \sin^2\frac{\theta_n}{2}) \hspace{1.0cm} \text{with }  n_0=\frac{2d}{\lambda}\\
	n = n_0\left(1-\left(\frac{\theta_n}{2}\right)^2\right) \Leftrightarrow \theta_n = \sqrt{\frac{2(n_0 - n)}{n_0}}
	\end{split}
\end{align}
for small angles $\theta$.
Because $n_0$ referring to equation~\ref{eq:neq1} is not an integer in general, the first maximum closest to the inner ring $n_1$ is given by 
\begin{align}
	\label{eq:neq2}
	\begin{split}
	n_1 	&= \left[n_0-\epsilon\right] \hspace{1.0cm} \text{with } n_1 \widehat{=} \text{ closest integer to } n_0 - \epsilon \text{ , or}\\
	n_p 	&=n_1 - p + 1 \ ,
	\end{split}
\end{align}
respecting to enumberate the maxima beginning at the center ring of the intensity spectrum.
This equation can be simplified using equation~\ref{eq:FocussalRadius}.
\begin{align}
	\label{eq:RSqLambda}
	\begin{split}
	r_p			&= \sqrt{\frac{2f}{n_0} \cdot \left((p-1) + \epsilon\right)}\\
	r_{p+1}^2-r_{p}^2 	&= \frac{2f}{n_0}
	\end{split}
\end{align}
To determine the wavelength, $n_0$ (therefore $\lambda$ in eq.~\ref{eq:neq1}) is given by the slope of a $r_p^2$ versus $p$ plot.
It is also possible to determine the parameter $\epsilon$ when this fit is extrapolated to\newline$r_p^2 = 0$.
With respect to that one can calculate the order of the first maximum, using relation~\ref{eq:neq2}.
If there is a splitting of the wavelengths, one can seperate $\epsilon_{\sigma^{\pm}}$ from equation~\ref{eq:neq2}
\begin{align}
	\begin{split}
	\epsilon_{\sigma^+} 	&= 2d \cdot \nu_{\sigma^+} - n_{{\sigma^+},i}\\
	\epsilon_{\sigma^-}	&= 2d \cdot \nu_{\sigma^-} - n_{{\sigma^-},i} \hspace{1.0cm} \text{with } \nu_{{\sigma^+},{\sigma^-}} = \frac{1}{\lambda_{{\sigma^+},{\sigma^-}}}
	\end{split}
\end{align}
Considering the order of the two rings $\nu_{{\sigma^+}{\sigma^-},i}$ as the same, the wavelength difference of $\left|\nu_{\sigma^+}-\nu_{\sigma^-}\right| =: \delta \nu$ is resolved to
\begin{align}
	\delta \nu	&= \frac{\epsilon_{\sigma^+} - \epsilon_{\sigma^-}}{2d}\ .
\end{align}
Equations~\ref{eq:RSqLambda} allow to eliminate $\epsilon$ and yield
\begin{align}
	\label{eq:NuOrder}
	\begin{split}
	\delta \nu_p 	&= \frac{1}{2d} \cdot \left(\frac{r_{{\sigma^+},p+1}^2}{r_{{\sigma^+},p+1}^2-r_{{\sigma^+},p}^2} - \frac{r_{{\sigma^-},p+1}^2}{r_{{\sigma^-},p+1}^2-r_{{\sigma^-},p}^2}\right)\\
			&= \frac{1}{2d} \cdot \frac{\xi^{p+1}}{\eta^{p+1,p}}\ ,
	\end{split}
\end{align}
because $r^2_{{\sigma^+},p+1} - r^2_{{\sigma^-},p+1}$ and $r^2_{{\sigma^+},p+1} - r^2_{{\sigma^+},p} = r^2_{{\sigma^-},p+1} - r^2_{{\sigma^-},p}$ have to be the same for every $p$.
The average of $\xi^{p+1}$ and $\eta^{p+1,p}$ then leads to an average of $\delta\nu_p$.\\
\\
According to equation~\ref{eq:ZeemanEnergyShift}, the energy shift is denoted to
\begin{align}
\begin{split}
	\delta E 	&= h\cdot f\\
			&= h\cdot c\frac{\overline{\delta\nu}}{2} \hspace{1.0cm} \Leftrightarrow \hspace{1.0cm} \mu_\text{B} = h \cdot c \frac{\overline{\delta\nu}}{2B}\ ,
\end{split}
	\label{eq:BohrMag1}
\end{align}
with an error of
\begin{align}
	\Delta \mu_\text{B} 	&= hc \cdot \frac{\overline{\delta\nu}}{2B}\sqrt{\left(\frac{\Delta \overline{\delta\nu}}{\overline{\delta\nu}} \right)^2+\left(\frac{\Delta B}{B} \right)^2}\ .
	\label{eq:BohrMag2}
\end{align}
