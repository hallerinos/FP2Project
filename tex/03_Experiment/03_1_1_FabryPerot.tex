\subsubsection{The Fabry-P\'erot Interferometer}
\label{toc:FabryPerot}
The Fabry-P\'erot interferometer describes an optical resonator, which consists of two parallel, partially transmitting mirrors. 
This interferometer only convey light that meets the resonance condition. It is reflected several times, but each time part of the light is transmitted. 
This results in many parallel beams, that are focused by a lense in order to achieve interference in the focal point. 
	\begin{figure}[ht]
		\centering
		\includegraphics[width=0.7\textwidth]{Images/FabryPerot.jpg}
		\caption[Schematic of the Fabry-P\'erot Interferometer]{Schematic of the Fabry-P\'erot Interferometer \cite{FabryPerot}}
		\label{fig:FabryPerot}
	\end{figure}\\
The path difference between two beams of the interferometer is given by
	\begin{align}
		\Delta = 2d\cos(\theta)\ ,
	\end{align}
which has to be an integer value of the light's wavelength in order to achieve constructive interference. 
Hence one gets a relation between the wavelength $\lambda$ and $\Delta$ which follows
	\begin{align}
		2d\cos{\theta_n} = m\lambda \hspace{1.0cm} \text{with } n \in \mathbb{N}\ .
		\label{eq:ConstructiveInterferenceFabry}
	\end{align}
Because of the cylindrical symmetry the Fabry-P\'erot interferometer forms interference circles in the focal plane, whose radii depend on the angulars $\theta_n$ and the focal length $f$. 
For inner rings one can use the small-angle approximation
	\begin{align}
		\label{eq:FocussalRadius}
 		r_n = \tan(\theta_n)f \approx \theta_n f\ .
	\end{align}
